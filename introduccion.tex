\chapter{Introducci\'on} % (fold)
\label{cha:introduccion}

  La ciudad cuenta con lugares turísticos o de interés y es fácil perderse al no 
  tener conocimiento del nombre de las calles, movilidades que circulan por la 
  zona o directamente no saber a dónde ir, un turista que viene por primera vez 
  tiene el deseo de conocer un restaurant criollo   o el comprador que necesita 
  localizar el cajero automático mas cercano, toda esta información es difícil de 
  conocer o conseguir ya que la gente del lugar no conoce el destino deseado o  
  pueden existir muchas razones por lo cual es fácil perderse y  el tiempo es uno 
  de los recursos  más preciados en la actualidad.\\

  Los lugares turísticos son de especial interés por parte de gente que viene a 
  conocer la ciudad y es una gran fuente de ingreso por parte de gente del lugar 
  como también de empresas que se dedican al turismo, lo más importante el 
  turista es contar con información confiable del lugar que va a visitar, esta 
  información es un gran aliciente para el crecimiento de esta industria.
   % (sin chimeneas).

  \section{Antecedentes} % (fold)
  \label{sec:antecedentes}
    Actualmente existen blogs o redes sociales dedicadas al intercambio de 
    información, pero  la gran mayoría se basa solamente en intercambiar opciones 
    o gustos sobre determinados lugares, o se centran exclusivamente en los 
    lugares de más alta afluencia turística por ejemplo: el salar de Uyuni, el 
    lago Titikaka o por fechas festivas por ejemplo el carnaval de Oruro.\\ 

    Alguno de estos portales son  exploroo.com, dopplr.com, tripwolf.com, que 
    ofrecen servicios como ser estado del tiempo, búsqueda de hoteles, con 
    opciones como poder programar una guía turística personalizada, pero con la 
    desventaja que para nuestro país la información es tan escasa como ambigua, 
    estos portales son de gran utilidad para llegar al destino pero no existe la 
    información adecuada para los usuarios que quieran moverse  y conocer la ciudad.
  % section antecedentes (end)

  \section{Descripción del problema} % (fold)
  \label{sec:desc_probl}
    En una ciudad o región donde existen muchos lugares de interés para propios o 
    extraños siempre falta una guía actualizada que nos permita movernos o tomar 
    decisiones, las guías disponibles son generalmente impresas y solo llevan un 
    registro de los lugares más destacados o de aquellos que pagan el derecho de 
    aparecer en la guía, además que no se tiene la certeza si la información  es 
    actual. \\

    En una ciudad los cambios se dan de un día para otro, locales que cierran, 
    calles que cambian de dirección,  rutas de movilidades (trufis, taxitrufis) 
    que son modificadas o creadas, donde tener información actual es primordial.\\

    Actualmente la Universidad Mayor de San Simon, no cuenta con un mapa
    actualizado de las locaciones  internas, y para los estudiantes nuevos o 
    personas que necesitan hacer algun tramite, se hace difícil el movilizarse
    dentro del Campus Universitario. 
    % por lo tanto es muy importante el contar 
    % con algun tipo de guia que nos permita recorrer y visitar el campus universitario.
  % section desc_probl (end)

  \section{Objetivo general} % (fold)
  \label{sec:objetivo_general}
    \begin{quote}
      Construir un Prototipo de Red Social para localización de lugares con 
      tecnología Ruby on Rails (RoR).
    \end{quote}
  % section objetivo_general (end)

  \section{Objetivos Específicos} % (fold)
  \label{sec:obj_especificos}
    \begin{itemize}
      \item Analizar el framework Ruby on Rails para el desarrollo de aplicaciones web2.0
      \item Proponer 3 patrones web2.0 aplicables al problema.
      \item Implementar 3 Patrones web2.0 con tecnología RoR.
      \item Analizar y construir un modulo con características de geo-localización para la Red Social.
    \end{itemize}
  % section obj_especificos (end)

  \section{Justificación:} % (fold)
  \label{sec:justificacion}
    La gran acogida de la población en general por las redes sociales, hace 
    hincapié en el desarrollo de un sistema de  que sea ágil y eficiente en el 
    manejo de información, ayudándose en las tecnologías  presentes en la web2.0 y 
    que tenga una gran cobertura, mediante el cual  ayude a conocer la ciudad  o 
    decidir al lugar donde nos podemos dirigir indicando una ruta optima.

  % section justificacion (end)
% chapter introduccion (end)


  
  
  




% Metodología:
% Agile Unified Process (AUP) es una versión simplificada de Rational  Proceso 
% Unificado  (RUP). 

% Fases del ciclo de desarrollo
%   Principio: El objetivo es  identificar el alcance inicial del proyecto y una 
%   arquitectura potencial del sistema.
  
%   Elaboración: El objetivo es confirmar la idoneidad de la arquitectura del sistema.
%   Construcción: El objetivo es desarrollar software funcional dentro de un sistema regular e incremental periódicamente que mire las necesidades de las partes interesadas.
%   Transición: El objetivo es validar y desplegar el sistema en su entorno de producción.
% Las disciplinas del ciclo de desarrollo se llevan de manera iterativa y son 
% las siguientes
%   Modelo: El objetivo de esta disciplina es entender el negocio de la organización, el dominio del problema que se ocupa el proyecto, y determinar una solución viable para hacer frente al dominio del problema.
%   Aplicación: El objetivo de esta disciplina es transformar el modelo de su (s) en el código ejecutable y para llevar a cabo un nivel básico de las pruebas, en las pruebas de unidad en particular.
%   Prueba: El objetivo de esta disciplina consiste en realizar una evaluación objetiva para asegurar la calidad. Esto incluye encontrar defectos, validar que el sistema funcione como está previsto, y verificar que se cumplan los requisitos.
%   Implementación: El objetivo de esta disciplina es el plan para la entrega del sistema y para ejecutar el plan para que el sistema a disposición de los usuarios finales.
%   Gestión de la Configuración: El objetivo de esta disciplina consiste en administrar el acceso a artefactos de su proyecto. Esto incluye no sólo el seguimiento de versiones de los artefactos a través del tiempo, sino también el control y la gestión de los cambios a los mismos.
%   Gestión de Proyectos: El objetivo de esta disciplina es dirigir las actividades que lleva a cabo en el proyecto. Esto incluye la gestión de riesgos, la dirección de personas (la asignación de tareas, seguimiento de los progresos, etc.), y coordinar con la gente y los sistemas fuera del alcance del proyecto para asegurarse de que se entregue a tiempo y dentro del presupuesto.
%   Para el Medio Ambiente: El objetivo de esta disciplina es apoyar el resto de los esfuerzos por garantizar que el proceso, la orientación adecuada (las normas y directrices), y herramientas (hardware, software, etc.) están disponibles para el equipo según sea necesario.
% % Fig. Ciclo de vida del Proceso Unificado Ágil 