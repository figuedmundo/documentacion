\chapter{Ruta Optima} % (fold)
\label{cha:ruta_optima}
  

  Si se quiere ir de un punto a otro el mejor camino siempre es el mas optimo, pero como se define que un camino sea óptimo, si se va en coche hay que tomar en cuenta la dirección de las calles, los cruces, etc. si se va a pie hay que ver las características del terreno, caminos cortados, distancias, etc. 


  Si se analiza el terreno que se va a cubrir con la aplicación (el campus de la UMSS), se tiene que el  camino optimo es siempre el más corto o de menor longitud. 

  La resolución de este problema es la se analizará en este capítulo.

  \section{Grafos} % (fold)
  \label{sec:teoria_grafos}
  
    El problema es encontrar la ruta más corta de un punto a otro punto, en donde los puntos están interconectados por una red de caminos. El problema descrito se lo puede resolver/describir como un caso espec\'ifico de la teoría de grafos.


    \subsection{Definiciones} % (fold)
    \label{sub:grafos_definiciones}
      Primeramente se aclarar\'an alguno términos usados en la teoría de grafos.\\

      % El grafo que es la reprentacion 

      % \begin{description}
      %   \item[Grafo] Un grafo G consiste en un conjunto  vértices V y un conjunto de aristas A, y se lo escribe como G(V,E).
      %   \item[Vertice] 
      % \end{description}
      
      Un \textbf{grafo} $G$ consiste en un conjunto de vértices $V$ y un conjunto de aristas $A$, y se lo representa con $G(V,A)$.\\ 

      El \textbf{vértice} \emph{v} es adyacente a \emph{u}, o a un vecino de \emph{u}, si y sólo si $(u,v) \in A$. Por lo tanto, en un grafo no dirigido, dado una arista $(u,v)$, $v$ es adyacente de $u$, y simétricamente \emph{u} es adyacente de \emph{v}. 
      Los vértices también son llamados nodos. \\

      Cada \textbf{arista} o arco es representada por un par de elementos $(u,v)$, donde los elementos $u,v \in V$, son los nodos que une la arista.
      En un grafo no dirigido el par de vértices que representan la arista no tiene orden, por lo tanto la arista $(u,v)$ y $(v,u)$ representa la misma arista. En cambio en un grafo dirigido la arista $(u,v)$ y $(v,u)$ representan dos diferentes aristas. También se puede anotar un tercer componente, llamado peso o costo, en ese caso estaríamos hablando de un \emph{grafo ponderado}.\\
      % Para fines prácticos, no  consideraremos las aristas de la forma (u,u)

      En un grafo no dirigido $G$, dos vértices $u$ y $v$ se dice que están conectados si hay un camino en $G$ de $u$ a $v$ (y como $G$ no es dirigido, también hay un camino de $v$ a $u$). Un grafo  se denomina completo si para todos los pares $u,v \in V$ existe una arista $(u,v) \in A$.\\

      Un camino en un grafo es una secuencia de nodos $v_{1}$, $v_{2}$, \ldots{}, $v_n$ tal que $(v_{1}, v_{2}), (v_{2}, v_{3}), \ldots{}, (v_{n-1}, v_n)$ son aristas. 
    % \end{description}
    % subsection grafos_definiciones (end)
    \subsection{Representacion de un Grafo} % (fold)
    \label{sub:representacion_de_un_grafo}
      Existen diversas formas de representar un grafo sea dirigido o no-dirigido, pero entre las mas usadas están la matriz de adyacencias y la lista de adyacencias.
      \subsubsection{Matriz de adyacencias de un Grafo} % (fold)
      \label{ssub:matriz_de_adyacencias_de_un_grafo}  
        Sea $G = (V,A)$ un grafo de \emph{n} vértices. La matriz de adyacencias $M$  para $G$ es una matriz $M_{nxn}$ de valores booleanos, donde $M(i,j)$ es verdad si y sólo si existe un arco desde el nodo \emph{i} al nodo \emph{j}.

        \begin{displaymath}
          M(i,j) = \left\{ 
          \begin{array}{ l l }
            1, & \textrm{si existe la arista } (i,j) \\
            0, & \textrm{en caso contrario}
          \end{array} \right.
        \end{displaymath}


        Las filas y las columnas de la matriz representan los nodos del grafo.
        Cuando el grafo no es dirigido la matriz de adyacencias es simétrica.
        % El cuadro \ref{tab:matriz} representa la matriz de adyacencias de la figura \ref{fig:grafo_ponderado} representa 
        La matriz de adyacencias, que se puede observar en el cuadro \ref{tab:matriz}, es la  misma matriz de la relación $A$ de $V$ en $V$ porque indica cuales v\'ertices están relacionados (unidos por una arista)


        \begin{figure}[!ht]
          \begin{center}

            \begin{tikzpicture}[->,>=stealth',shorten >=1pt,auto,node distance=3cm,
                    main node/.style={circle,draw,font=\sffamily\Large\bfseries}]
                    
              \node[main node] (1) {a};
              \node[main node] (2) [below right  of=1] {b};
              \node[main node] (3) [above right of=2] {c};
              \node[main node] (4) [below left of=2] {d};
              \node[main node] (5) [right of=2] {e};
              % \node[main node] (6) [right of=5] {f};
              % \node[main node] (7) [above right of=6] {g};
              % \node[main node] (4) [below right of=1] {d};

              \path[every node/.style={font=\sffamily\small}]
                (1) edge node [auto] {3} (2)
                    edge node[left] {5} (4)
                (2) edge node[left] {8} (3)
                    edge node[right] {4} (4)
                    edge node[auto] {3} (5)
                (3) edge node[left] {7} (5)
                (4) edge node[below] {14} (5);
                
            \end{tikzpicture}

          \end{center}
          \caption{Grafo ponderado no-dirigido}
          \label{fig:grafo_ponderado}
        \end{figure}

        \begin{table}[!ht]
          \label{tab:matriz}
          \begin{center}
            \begin{displaymath}
              M(i,j) =
              \bordermatrix{ ~ & a & b & c & d & e \cr
                             a & 0 & 3 & 0 & 5 & 0 \cr
                             b & 3 & 0 & 8 & 4 & 3 \cr
                             c & 0 & 8 & 0 & 0 & 7 \cr
                             d & 5 & 4 & 0 & 0 & 14\cr
                             e & 0 & 3 & 7 & 14& 0  }
            \end{displaymath}
            \caption{Matriz de adyacencias del grafo de la figura  \ref{fig:grafo_ponderado}}
          \end{center}
        \end{table}


      % subsubsection matriz_de_adyacencias_de_un_grafo (end)
      
    % subsection representacion_de_un_grafo (end)
    \subsection{Ruta mas corta} % (fold)
    \label{sub:ruta_mas_corta}
      Dados los vértices $v_{i}$ y $v_{j}$ de un grafo $G = (V,A)$ se llama trayectoria mínima o camino minimo  de \(v_i\) a \(v_j\) al numero de aristas del camino de longitud mínima que va desde $v_i$ a $v_j$ y se representa por $d(v_i, v_j)$.

      Cuando en el grafo no exista un camino de $v_i$ a $v_j$ se dice que el camino minimo es $d(v_i, v_j) = \infty$ \\

      Para determinar el camino mínimo que va desde un único vértice a cualquier otro vértice se puede usar el algoritmo de Dijkstra. 
      


      \subsubsection{Algoritmo de Dijkstra} % (fold)
      \label{sub:algoritmo_de_dijkstra}
      El algoritmo de  Dijkstra fue descrito en 1959 por Edsger Dijkstra, y permite encontrar la trayectoria más corta entre dos nodos específicos, cuando los valores de los arcos son todos positivos\\

      El algoritmo asigna un etiqueta a cada nodo en el grafo. Esta etiqueta es la distancia que hay desde el nodo \emph{s} escogido como origen a lo largo de la trayectoria más corta encontrada, hasta el nodo que se está etiquetando.\\

      La etiqueta de cada nodo puede estar en 2 estados:

      \begin{itemize}
        \item[a.] Puede ser permanente: en este caso la distancia encontrada es a lo largo de la trayectoria más corta de todas las encontradas.
        \item[b.] Puede ser temporal: cuando hay incertidumbre de que la trayectoria encontrada sea la más corta de todas.
      \end{itemize}

      A medida que el método trabaja se cambian gradualmente las etiquetas temporales por etiquetas permanentes. Al comienzo se tiene un conjunto de nodos con etiquetas temporales y el objetivo es hacer que esas etiquetas disminuyan, encontrando trayectorias a esos nodos usando trayectorias a nodos etiquetados permanentemente. Cuando esto se ha logrado, se selecciona el nodo con la etiqueta temporal más pequeña y esta etiqueta se convierte en permanente. El proceso se repite hasta que al nodo terminal \emph{t} se le haya asignado una etiqueta permanente, pero esto puede ocurrir eventualmente, ya que cada vez que el algoritmo es usado, una de las etiquetas es omitida y así el número de nodos con etiquetas temporales decrece a cero. \cite{teoria_grafos} \\

      
      % subsection algoritmo_de_dijkstra (end)
    % subsection ruta_mas_corta (end)
  % section teoria_grafos (end)
  
  \section{Conclusion} % (fold)
  \label{sec:ruta_conclusion}

    Existen numerosas soluciones para encontrar la ruta óptima, donde se toman en cuenta diferentes variables y heurísticas, en el que  cada algoritmo presenta ventajas respecto a las demás.
    La teoría de grafos  es un tema extenso y para fines prácticos 
    solo se explicó el algoritmo de Dijkstra por ser el que se esta usando en la aplicación desarrollada.\\
    
    El algoritmo de Dijkstra puede ser una de las soluciones más sencillas y que requiere muchos más cálculos que las demás pero el grafo implementado al no ser extenso, no existe una razón de mucho peso para buscar otra solución más eficiente en el manejo de recursos.

  % section ruta_conclusion (end)
% chapter ruta_optima (end)


  % Algoritmos de busqueda de caminos - ruta corta

  % Como determino que tipo de grafo tengo


  % % \section{La Red} % (fold)
  % % \label{sec:la_red}
  
  % % % section la_red (end)

  % \section{Algoritmo} % (fold)
  % \label{sec:algoritmo}
  
  % % section algoritmo (end)

  % \section{Algoritmo de Dijkstra} % (fold)
  % \label{sec:algoritmo_de_dijkstra}
  
  % % section algoritmo_de_dijkstra (end)
  
  % % Por lo tanto se implementó un grafo no dirigido (sin dirección), el cual se analizará en este capítulo.


  % Un problema de este tipo es represetable como un proble de teoria de grafos.


  % Cuando se tiene que encontrar un camino o ruta optima entre 2 puntos, se tienen que tomar en cuenta varios puntos


  % El problema de encontrar una ruta optima entre 2 puntos se lo puede resolver/representar como problema de grafos


  % Caminos mínimos en grafos

  % Solución voraz: Algoritmo de Dijkstra

  % para grafos dirigidos (la extensión a no dirigidos es inmediata)
  % genera uno a uno los caminos de un nodo v al resto por orden creciente de longitud
  % usa un conjunto de vértices donde, a cada paso, se guardan los nodos para los que ya se sabe el camino mínimo
  % devuelve un vector indexado por vértices: en cada posición w se guarda el coste del camino mínimo que conecta v con w
  % cada vez que se incorpora un nodo a la solución se comprueba si los caminos todavía no definitivos se pueden acortar pasando por él
  % se supone que el camino mínimo de un nodo a sí mismo tiene coste nulo
  % un valor en la posición w del vector indica que no hay ningún camino desde v a w
  % E.W. Dijkstra:
  % “A note on two problems in connexion with graphs”,
  % Numerical Mathematica, 1, pp. 269-271, 1959.
