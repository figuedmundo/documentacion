\chapter{Ruby on Rails y patrones Web 2.0} % (fold)
\label{cha:ruby_on_rails_y_patrones_web_2_0}

  \section{Porque usar Ruby on Rails para desarrollar una aplicación web ?} % (fold)
  \label{sec:porque_usar_ruby_on_rails_para_desarrollar_una_aplicacion_web}

    La gran propaganda de Ruby on Rails (RoR) o más sencillamente Rails
    se basa en el rápido desarrollo de aplicaciones web conocido como agile development.

    Ruby on Rails maneja las filosofías DRY\footnote{Don’t Repeat Yourself} y convención sobre configuración. 

    \begin{description}
      \item[No te repitas(DRY)] según el creador de RoR, David Heinemeier Hansson, 
      significa que cada pieza de conocimiento en un sistema 
      debe ser declarado en un solo lugar.\cite{awdr4e} 
      Esto se logra gracias al patrón MVC\footnote{Model, View, Controller} 
      y el lenguaje multiparadigma Ruby sobre los cuales está construido Ruby on Rails.\\
      
      \item[Convención sobre configuración] significa que Rails tiene parámetros por 
      defecto para casi todos los aspectos que mantiene unida una aplicación, 
      siguiendo  las convenciones de Rails se llega a simplificar el código escrito en una aplicación.\\
    \end{description}
      % paragraph dry (end)


    David Heinemeier cita en su libro \cite{awdr4e}, que \emph{“Rails es Ágil porque 
    simplemente la agilidad es parte de su construcción”}.\\

    Los principios del manifiesto por el desarrollo ágil de software:
    \begin{itemize}
      \item \textbf{Individuos e interacciones} sobre procesos y herramientas
      \item \textbf{Software funcionando} sobre documentación extensiva
      \item \textbf{Colaboración con el cliente} sobre negociación contractual
      \item \textbf{Respuesta ante el cambio} sobre seguir un plan
    \end{itemize}
    
    Rails se enfoca bastante en conseguir un prototipo funcional en muy poco tiempo 
    y sobre ese prototipo seguir incrementalmente hasta conseguir una aplicacion 
    de calidad en poco tiempo.\\

    Los más grandes obstáculos que se enfrenta una aplicación en el tiempo es el 
    mantenimiento y escalabilidad, actualmente se estima que existen 230,000 websites\cite{web2} 
    desarrolladas sobre RoR\footnote{Ruby on Rails, tambien se lo puede nombrar \emph{Rails}, \emph{RoR}}, 
    entre ellas se puede nombrar a GitHub, Hulu, Yellow Pages. Son sitios con miles de visitas
    con una alta carga del servidor y son un claro ejemplo de que Rails puede manejar sitios de alto perfil.\\

    Los detractores de Rails sostienen que escalar una aplicación construida 
    sobre RoR es muy difícil pero los defensores argumentan que lo que se 
    tiene que escalar es el código de la aplicación no el framework.\\

    Twitter nació sobre Ruby on  Rails  y no fue hasta que era un servicio usado 
    a nivel mundial y manejaba millones de request por día que empezaron a surgir 
    problemas debido a que Ruby no estaba optimizado para un trabajo muy pesado, 
    según palabras de Alex Payne, Twitter developer, “Ruby es lento”\cite{web3}. 
    Actualmente Twitter migró su backend a Scala, framework basado en Java(que esta mas optimizado que Ruby),
    pero para su front-end no cambian a Rails.\cite{web4}\\
    
    Se puede agregar que Rails es una muy buena opción a la hora de empezar 
    cualquier proyecto web, ya que implementa las herramientas necesarias 
    para un desarrollo ágil, sólido y de calidad respaldado por un modelo 
    de desarrollo basado en pruebas(TDD), las filosofías DRY y convención sobre configuración.
    y cundo la aplicaci\'on haya crecido y empiecen a aparecer muchos problemas es 
    decisión de los programadores el ver si mantener el código actual y parchearlo o 
    cambiar de tecnología para mejorar el rendimiento y la experiencia del usuario\\
  % section porque_usar_ruby_on_rails_para_desarrollar_una_aplicaci_n_web_ (end)

  \section{Patrones de diseño de la Web2.0} % (fold)
  \label{sec:patrones_web20}
    Que es la Web2.0 ?,  primeramente  se debe explicar que este término fue acuñado 
    por 1999 para describir paginas web que usaban tecnologias mas alla de las 
    simples estaticas paginas web.\\
    No fue hasta que en  el 2004 en la conferencian sobre la Web2.0 que popularizo este termino, 
    y asi mismo como la Web que evoluciona, la definicion se actualiza con el tiempo, 
    pero Tim O’Reilly trato de definirla en su articulo “What is Web 2.0”\cite{web5}, 
    del que se puede extraer la siguiente definici\'on :
    \begin{quote}
      “Web 2.0 is the business revolution in the computer industry caused by the move to the
       Internet as a platform, and an attempt to understand the rules for success on that new
       platform. Chief among those rules is this: build applications that harness network
       effects to get better the more people use them.”
       \begin{flushright}
       --Tim O’Reilly \cite{web9}
       \end{flushright}
    \end{quote}
    En resumen se puede definir que una aplicación web2.0 es aquella que mejora y crece con la 
    participación activa de sus usuarios.

    \begin{quote}
      “Software que mejora mientras más gente la usa” \cite{web5}
    \end{quote}

    Un patrón de diseño es una solución general, reusable  y flexible 
    que describe cómo resolver algún problema general en el desarrollo 
    de software, un patrón puede ser usado y modificado segun el problema 
    al cual se esta aplicando.\\
    Se pueden observar los siguientes patrones de dise\~no en la aplicación:
    \begin{itemize}
      \item REST
      \item MVC
      \item Mashup
    \end{itemize}
    
  % section patrones_web20 (end)


% chapter ruby_on_rails_y_patrones_web_2_0 (end)