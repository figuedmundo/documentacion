\chapter{Marco Referncial} % (fold)
\label{cha:marco_referncial}

  % \section{Introducci\'on} % (fold)
  % \label{sec:Introduccion}
    
  % % section Introduccion (end)

  \section{Ruby on Rails} % (fold)
  \label{sec:ruby_on_rails}
    Ruby on Rails es un framework dise\~nado para desarrollar apliaciones web, 
    y esta contruido sobre el lenguaje de programacion Ruby, Ruby  fue creado
    alrededor de 1993 por Yukihiro ``Matz'' Matsumuto de Japon, y liberado al 
    publico en 1995, y desde entonces fue ganando en popularidad y 
    reputaci\'on gracias al aporte de una gran variedad de programadores, que 
    realzan la sintaxis elegante y el codigo limpio que se genera, Ruby es un 
    lenguaje de programacion multiparadigma ya que implementa programaci\'on 
    Orientado a Objetos, programaci\'on Funcional asi como tambien 
    programaci\'on Imperativa. \\

    Ruby on Rails fue creado en el 2004 por David Heinemeier Hansson durante el desarrollo de Basecamp, una aplicación de gesti\'on de proyectos, y una ves que se necesito para otros proyectos, el equipo de desarrollo extrajo el core de funcionalidad el cual fue presentado al publico en julio del 2004 con el nombre de Ruby on Rails, como proyecto Open Source bajo una licencia MIT, desde entonces tuvo un gran  crecimiento impulsado por la comunidad de usuarios que continuamente estan desarrollando nuevas caracteristicas, limpiando bugs y creando gemas\footnote{ Los plugins o complementos, en el lenguaje Ruby son llamados \textbf{gemas}}. La ultima version de Rails es la 3.2 publicado en enero del 2012 y actualmente esta en su revisi\'on 3.2.8 presentado en agosto del 2012,  demostrando que el equipo de desarrollo de Rails esta trabajando constantemente en mejorar este framework que actualmente esta entre los mejores en desarrollo web.\\
     % y se prevee que la version 4 de Rails sea lanzada a finales del 2012, pero no  \\
    % y para el proyecto se utilizo la version 3.2.3

    En nucleo de funcionalidad de Rails es un conjunto de funciones llamadas \emph{Railties}:
    \begin{description}
      \item[Active Record] Es una implementaci\'on del patron
        Object-Relational Mapping(ORM), que mapea las tablas de la Base de datos relacional en clases, filas en objetos y 
        columnas en atributos de los objetos.   
      \item[Active Support] Es el componente de Rails responsable de proporcionar extensiones del lenguaje Ruby, utilitarios y funciones primordiales a la hora de realizar cualquier tarea en el desarrollo de la aplicaci\'on.  
      \item[Action Mailer] Permite enviar correo electronico (email) desde la aplicaci\'on usando un  modelo y vistas.
      \item[Action Pack] Es el responsable de manejar y responder los request del navegado web. Provee las herramientas para el \textbf{routing}, define los \textbf{controladores}, y genera las respuestas renderizando las \textbf{vistas}. En resumen, Action Pack maneja las capas de la vista y el controlador del paradigma MVC.  
    \end{description}
    Otra de las caracteristicas de Rails es la facilidad para escribir Pruebas, en realidad Rails sugiere el modelo de Desarrollo guiado por Pruebas(TDD\footnote{Test-Driven Development, por sus siglas en Ingles}), que consiste en 3 pasos
    \begin{enumerate}
      \item \textbf{Rojo}, la prueba falla
      \item \textbf{Verde}, la prueba pasa
      \item \textbf{Refactorizar}, limpiar el codigo 
    \end{enumerate}
    Para este proceso, Rails ofrece primeramente el modulo Test::Unit, pero tambien se pueden encontrar variadas herramientas para llevar a cabo esta tarea.\\

    Ruby on Rails es un framework MVC, que implementa los principios REST, No te Repitas\footnote{DRY, Don't Repit Yourself}, Convenci\'on sobre Configuraci\'on, estas caracteristicas de Rails estan explicadas con mayor detalle en el capitulo \ref{cha:ruby_on_rails_y_patrones_web_2_0}.


  % section ruby_on_rails (end)

  \section{Base de Datos} % (fold)
  \label{sec:base_de_datos}

    En una aplicacion web es necesario alguna forma de persistencia de datos, en especial si se estan usando datos complejos y en gran cantidad, para realizar esta tarea, la base de datos esfactor primoridal.
    Rails maneja la base de datos mediante  un ORM, por lo tanto la base de datos que se utilize no es tan excluyente, en este caso se utilizo  \emph{PostgreSQL} como base de datos relacional.\\

    \subsection{PostgreSQL} % (fold)
    \label{sec:postgres}

      PostgreSQL es un sistema de gestión de bases de datos objeto-relacional, Open Source y distribuido bajo licencia BSD. 
      PostgreSQL utiliza un modelo cliente/servidor y usa multiprocesos en vez de multihilos para garantizar la estabilidad del sistema. Un fallo en uno de los procesos no afectará el resto y el sistema continuará funcionando.
      La última versi\'on de PostgreSQL es la 9.2, su desarrollo comenzo hace más de 16 años, y cuenta con una gran comunidad que aporta con el desarrollo, testeo de nuevas versiones.
      PostgreSQL  esta considera como una de los mejores \emph{Sistemas de gesti\'on de bases de datos}, es muy completo y esta muy bien documentado\footnote{ http://www.postgresql.org/docs/9.2/static/}. 
      Entre sus caracteristicas se pueden nombrar las siguientes.
      \begin{itemize}
        \item Es una base de datos 100\% ACID\footnote{  ACID es un acrónimo de Atomicity, Consistency, Isolation and Durability}
        \item Integridad referencial
        \item Replicación asincrónica/sincrónica
        \item Multiples métodos de autentificación
        \item Disponible para Linux y UNIX en todas sus variantes
        \item Funciones/procedimientos almacenados
        \item Soporte a la especificaci\'on SQL
      \end{itemize}

      Personalmente se escogio trabajar con  PostgreSQL como DBMS
      porque cuenta con una extensa documentacion,  y gracias a su caracter ``Open Source'', y su gran flexibilidad en poder definir nuevos tipos de datos, 
      se hace posible que empresas como \textbf{Refractions Research} puedan crear recursos como PostGIS, necesario para trabajar con datos geograficos \'o espaciales.

      % Entre sus principales  caracteristicas se puede nombrar que es
      % \footnote{ DBMS, DataBase Management System}
      % y durante este tiempo, estabilidad, potencia, robustez, facilidad de administración e implementación de estándares han sido las características que más se han tenido en cuenta durante su desarrollo. PostgreSQL funciona muy bien con grandes cantidades de datos y una alta concurrencia de usuarios accediendo a la vez a el sistema.

    % section postgres (end)

    \subsection{PostGis} % (fold)
    \label{sec:postgis}

      PostGIS es un modulo  que a\~nade soporte de objetos geográficos al DBMS PostgreSQL, convirtiendola en una base de datos espacial para su utilizacion en un Sistema de Informaci\'on Geografica(SIG\footnote{ Es bastante comun utilizar el acronimo en Ingles, Geographic Information System (GIS), de hay viene el termino de PostGIS = Postgres + GIS}).

      El desarrollo de PostGIS esta a cargo de \textbf{Refractions Research}, esta liberada con la \emph{Licencia pública general de GNU}, declarandola como software libre y lo protege de cualquier intento de apropiaci\'on.\\

      PostGIS implementa la especificaci\'on ``SFSQL'' (Simple Features for SQL, define los tipos y funciones que necesita implementar cualquier base de datos espacial) de la OGC (Open Geospatial Consortium, es un consorcio internacional, formado por un conjunto de empresas, agencias gubernamentales y universidades, dedicado a desarrollar especificaciones de interfaces para promover y facilitar el uso global de la información espacial).\\

      PostGIS al igual que PostgreSQL tiene una documentaci\'on bastante extensa, y cuenta con equipo de desarrollo que continuamente va sacando nuevas versiones, actualmente se encuentra la versi\'on 2.0.1, pero para el desarrollo de la aplicaci\'on se hizo uso de la versi\'on 1.5.5.

      PostGIS es gratis, pero no por ello es una herramienta de baja calidad, al contrario se la considera una herramienta de nivel empresarial, y muchas instituciones la est\'an usando de manera exitosa\footnote{ http://www.postgis.org/documentation/casestudies/}, aparte de numerosas aplicaciones \\

      Manejar los datos geograficos con PostGIS es sencillo y muy eficiente, por esta raz\'on se utilizo esta herramienta, pero para conseguir la ruta optima entre 2 puntos se necesitaba el uso del algoritmo de Dijkstra y para PostGIS existe un modulo \textbf{PgRouting}, que tiene implementado este algoritmo.
      
      \subsubsection{pgRouting} % (fold)
      \label{sec:pgrouting}
        pgRouting es una extensi\'on  de  PostGIS para proveer funcionalidades de ruteo espacial. pgRouting es un desarrollo posterior de pgDijkstra y actualmente esta siendo mantenido por Georepublic, la ultima versi\'on estable es la 1.05, y es la que fue usada para desarrollar el sistema.\\

        Las ventajas del ruteo en la base de datos son:
        \begin{itemize}
          \item Los datos y atributos pueden ser modificados desde varios clientes, como Quantum GIS y uDig a través de JDBC, ODBC, o directamente usando Pl/pgSQL. Los clientes pueden ser PCs o dispositivos móviles.
          \item Los cambios pueden ser reflejados instantáneamente a través del motor de ruteo. No hay necesidad de hacer cálculos previos.
          \item El parámetro de ``costo'' puede ser calculado dinámicamente a través de SQL y su valor puede provenir de múltiples campos y tablas.
        \end{itemize}

        pgRouting provee funciones para:
        \begin{itemize}
          \item Camino mínimo (Dijkstra): algoritmo de ruteo sin heurística
          \item Camino mínimo (A-Star): routeo para conjunto de datos grandes (con heurística)
          \item Camino mínimo (Shooting-Star): ruteo con restricciones de giro (con heurística)
          \item El problema del viajante (TSP: Traveling Salesperon Problem)
          \item Cálculo de ruta (Isolíneas)
        \end{itemize}
        
        % Uses PostGIS for its geographic data format, which in turn uses OGC’s data format Well Konwn Text (WKT) and Well Known Binary (WKB)
      % section pgrouting (end)
    % section postgis (end)
  % section base_de_datos (end)
% chapter marco_referncial (end)